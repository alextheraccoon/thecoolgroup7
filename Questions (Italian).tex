\documentclass[12pt]{article}
\usepackage{graphicx}
\usepackage{hyperref}

\title {
\includegraphics[width = .2\linewidth]{University-of-Lugano.png} \break \break
{\bf\Huge Interview Transcript}
\\\large Human-Computer Interaction
\\\small Academic Year 2017/2018 \break
\\\large \textbf{Group 7}
\\\large Alessandra Vicini \\ Edoardo Lunardi \\ Ozren Dabic \\ Pasquale Polverino \\ Paganoni Marco}
\date{March 05, 2018}


\begin{document}
\maketitle
\thispagestyle{empty}

\newpage
\setcounter{page}{1}
\noindent Interviewer: Edoardo Lunardi
\section*{Questions}
Ho intervistato 14 gruppi di ragazzi: 9 gruppi da 3 e 5 gruppi da 2. Totale ragazzi: 37.
\begin{enumerate}

  \item Sapete perche' sono qui oggi? \\
  La maggior parte dei ragazzini sapeva il motivo per il quale ero li.
  \item Come e quanto usate il vostro smartphone? \\
  Diciamo che il 95\% dei ragazzi possiede uno smartphone ed usa regolarmente Whatsapp ed Instagram. Qualcuno possiede un normale telefono cellulare e non utilizza alcun tipo di applicazione social o di messaggistica.
  Chiedendo loro quale fosse il rapporto genitore-smartphone e figlio, le risposte sono state divergenti: il 60\% circa ha la possibilta' di usare il proprio dispositivo in modo autonomo senza alcun tipo di controllo da parte
  dei genitori, mentre altri sono soggetti a controlli/limitazioni da parte di essi (i genitori vogliono sapere con chi si relazionano i figli e hanno il potere di "ritirare" lo smartphone come punizione). Il dispositivo viene
  utilizzato principalmente per comunicare con gli altri amici, soprattutto utilizzando Whatsapp nel gruppo relativo alla classe.
  \item Fate sport? Quali? \\
  Ho fortunatamente incontrato ragazzi spigliati e molto sportivi: tutti praticano sport. Gli sport praticati in questione sono:
  \begin{itemize}
    \item Calcio (30\%)
    \item Basket (40\%)
    \item Equitazione (5\%)
    \item Nuoto (5\%)
    \item Tennis (5\%)
    \item Hockey (10\%)
    \item Danza (3\%)
    \item Judo (2\%)
  \end{itemize}
  \newpage
  \item Quali sono i vostri hobby?
  I principali hobby registrati sono:
  \begin{itemize}
    \item Giocare ai videogiochi
    \item Recitazione/Teatro
    \item Studio di lingue straniere
    \item Leggere
    \item Canto
    \item Suonare la chitarra
    \item Ballare
  \end{itemize}
  \item Organizzate incontri con i vostri amici per attivita all'aperto? \\
  Chi pratica sport di squadra organizza anche incontri/partite con i propri amici nei campi in zona. Chi pratica sport individuali preferisce uscire con gli amici per incontrarsi a fare un giro in centro citta'.
  \item Vi fidereste a dare i vostri dati personali ad una applicazione come la
    nostra? Che problema di fondo potrebbe esserci in un'applicazione che organizza incontri tra ragazzi? \\
    Le opinioni si sono divise in due parte: un 65\% regolarmente iscritto a un social network come Instagram non ha alcun problema a fornire i propri dati personali, quali nome, cognome e una foto di profilo. Un 25\% preferirebbe fornire solo
    nome e cognome e il rimanente 10\%, iscritto su Instragram sotto un altro nome, preferisce non fornire alcun dato personale. La quasi totalita' di loro ha capito il grande problema di fondo, ovvero la verifica dell'identita'.
  \item Secondo voi c'e' un modo affidabile per verificare che il ragazzino che andrete a incontrare sia un vostro coetano? \\
    Senza che io dessi loro la soluzione che abbiamo pensato, qualcuno ha proposto di dividere in qualche modo le fasce d'eta' d'incontro. Ho detto loro che il modo piu' affidabile e' quello di appoggiarsi sulla scuola, la quale fornisce essa
    stessa i profili ai ragazzi. In questo modo soltanto chi ha un profilo fornito dalla scuola di appartenenza (e quindi verificato) puo' utilizzare l'applicazione. Hanno trovato la proposta intelligente, in quanto essere sicuri di chi si
    va a incontrare sembra essere stata la loro priorita' e soprattutto quella dei genitori.
  \item Secondo voi puo' funzionare? Vi sembra un modo affidabile? \\
  I ragazzi mi hanno risposto con grande entusiasmo alla proposta, soprattuto chi pratica uno sport di squadra. Dopo aver infatti detto loro come intendiamo risolvere il problema della verifica dell'identita'
  hanno dato un riscontro molto positivo alle proposte.
  \item L'ultima domanda che vorrei farvi e' la seguente: preferireste giocare al parco o al campo di calcio o quello che sia con un vostro coetaneo o con un ragazzo piu'
  grande/piu' piccolo di voi? \\
  Ho riscontrato pareri concordanti: la quasi totalita' ha espresso il parere di voler giocare con persone della propria eta' o con una differenza massima di $\pm$ 3 anni. Solo una ragazzina (che pratica basket da 7 anni) ha chiaramente espresso di aver piacere a giocare
  con persone molto piu' grandi di lei. Perche'? Per migliorare!
  \item Cosa ne pensate? Grazie per la disponibilita'. \\
  I ragazzi mi hanno proposto varie idee:
  \begin{itemize}
    \item Un sistema di feedback sia sul partecipante all'evento, sia sull'evento in questione. Ho quindi chiesto loro: e se diventasse uno strumento di diffamazione? Dopo tale domanda, circa un 60\% ha risposto che sarebbe meglio non mettere
    tale funzione, mentre il restante 40\% ha detto che sarebbe una feature utile solo se accompagnata da un commento di giustifica di tale feedback.
    \item Una chat tra persone che andranno a partecipare allo stesso evento, la quale viene "aperta" automaticamente quando tutti gli utente necessari hanno aderito all'evento.
  \end{itemize}
\end{enumerate}

\section*{Conclusion}

Ho cercato di stare il piu' imparziale possibile nel porre gli argomenti/domande agli alunni. Ovviamente in gruppi di due/tre persone c'e' sempre l'elemento "leader" che influenza le scelte/risposte
degli altri e qualcuno che si lascia influenzare. Ho cercato di entrare nell'intervista con degli story telling, lasciando ampio spazio ai ragazzi di dire la loro sin dalla prima domanda.

\end {document}
