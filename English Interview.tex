\documentclass[12pt]{article}
\usepackage{graphicx}
\usepackage{hyperref}

\title {
\includegraphics[width = .2\linewidth]{University-of-Lugano.png} \break \break
{\bf\Huge Interview Transcript}
\\\large Human-Computer Interaction
\\\small Academic Year 2017/2018 \break
\\\large \textbf{Group 7}
\\\large Alessandra Vicini \\ Edoardo Lunardi \\ Ozren Dabic \\ Pasquale Polverino \\ Paganoni Marco}
\date{March 07, 2018}


\begin{document}
\maketitle
\thispagestyle{empty}

\newpage
\setcounter{page}{1}
\noindent Interviewer: Edoardo Lunardi
\section*{Questions}
I interviewed 14 groups of children: 9 groups of 3 and 5 groups of 2. Total interviewees: 37.
\begin{enumerate}

  \item Hi guys, do you know why I'm here today? \\
  The majority of the children knew the reason why I was there.
  \item How much time do you spend using your smartphone, if you have one? \\
  95\% of the children have a smartphone and regularly use Whatsapp and/or Instagram. The other 5\% have a normal cellphone and don't use any kind of social/instant messaging applications.
  While asking about the relation between their smartphone and their parents, the answers differed: about 60\% have the possibility of using the smartphone independently without any restriction from their parents,
  whereas the remaining 40\% are subject to regulations and restrictions.
  The majority of the interviewees use smartphones for chatting with friends via Whatsapp.
  \item Do you play any sports? \\
  They all like sport, and all of them regularly engage in sports activities. These include:
  \begin{itemize}
    \item Football (30\%)
    \item Basketball (40\%)
    \item Horse riding (5\%)
    \item Swimming (5\%)
    \item Tennis (5\%)
    \item Hockey (10\%)
    \item Dancing (3\%)
    \item Judo (2\%)
  \end{itemize}
  \newpage
  \item What are your hobbies?
  Hobbies mentioned include:
  \begin{itemize}
    \item Playing video games
    \item Acting
    \item Studying foreign languages
    \item Reading books
    \item Singing
    \item Playing the guitar
    \item Dancing
  \end{itemize}
  \item Do you organise meetings with your friends outside? \\
  Those, who play team sports organise outside dime a dozen meetings and matches with their friends.
  Meanwhile, those who play individual sports prefer to go out and meet with friends and then engage in activities, such as going for a stroll in the centre of Lugano.
  \item Would you provide personal information to an application that organises meetings between children?
  In your opinion, is there any critical problem? \\
    Opinions are divided into two parts: about 65\% of the participants are on a social network like Instragram and do not have any problem to provide personal information, like name, surname and a profile picture. The 25\% would like to provide only the
    name and last name, while the remaining 10\% possess accounts on networks like Instragram, but under a fake name, all the while not being motivated to provide any kind of personal information. The majority of them understood the big ethical issue of identity verification.
  \item In your opinion, is there a way to verify that the child that you are going to meet has really your own age? \\
    Without giving them any hint, someone suggested to divide in some way the age ranges. I told them that the most efficient way to resolve this is to base on the school, which can provide the children's user profiles.
    In this way, only those who attend a specific school have a profile, which is provided and verified by the school itself. They found this proposal to be smart, because being sure that they are going to meet children of
    the same age (more or less) seems to be of highest priority (and of highest priority for their parents).

  \item Can it work? \\
  They were excited and convinced about the proposal, especially because they can be sure about the age verification. \\
  \item Would you prefer to play at the park or at the football field or whatever with a child with exactly your own age or you may want also children younger/older than you? \\
  Almost all of them want to play with children of the same age or with children that are maximum 3 years youger/older that them.
  Only one of them clearly stated that she would play with older people. Why? To improve!. \\
  \item In the end, what do you think? \\
  The children gave me some new ideas:
  \begin{itemize}
    \item A feedback system about the participants attending the event and also about the event itself. Thus, I asked: what if this became a defamation tool? After this, about the 60\% of them told me that it would be better not to implement the feedback feature,
    whereas the other 40\% said that it would be a cool and useful feature only if the feedback is justified by an appropriate comment.
    \item A chat between people who are going to participate at the same event, which will be automatically opened when all the users subscribed to the event.
  \end{itemize}
\end{enumerate}

\end {document}
