\documentclass[12pt]{article}
\usepackage{graphicx}
\usepackage{hyperref}

\title {
\includegraphics[width = .2\linewidth]{University-of-Lugano.png} \break \break
{\bf\Huge Interview Transcript}
\\\large Human-Computer Interaction
\\\small Academic Year 2017/2018 \break
\\\large \textbf{Group 7}
\\\large Alessandra Vicini \\ Edoardo Lunardi \\ Ozren Dabic \\ Pasquale Polverino \\ Paganoni Marco}
\date{March 07, 2018}


\begin{document}
\maketitle
\thispagestyle{empty}

\newpage
\setcounter{page}{1}
\noindent Interviewer: Edoardo Lunardi
\section*{Questions}
I interviewed 14 groups of children: 9 groups of 3 and 5 groups of 2. Total children: 37.
\begin{enumerate}

  \item Hi guys, do you know why I'm here today? \\
  The majority of the children knew the reason why I was there
  \item How and how much time do you use your smartphone, if you have one? \\
  95\% of the children has a smartphone and regularly use it for Whatsapp and Instagram. The other 5\% has a normal cellphone and don't use any kind of social/instant messaging application.
  While asking about the relation between them, their smartphone and their parents, the answers were different: about 60\% has the possibility of using the smartphone independentely without any restriction from the parents,
  whereas the remaining 40\% are subject to regulation (for instance, parents want to know with who they talk using the smartphone and they have the power to take it away as a punishment).
  The device is mainly used for chatting with friends using the Whatsapp group of the respective class.
  \item Do you play any sports? \\
  Luckily, they like sport so much: all of them regularly play some sports. They are:
  \begin{itemize}
    \item Football (30\%)
    \item Basketball (40\%)
    \item Horse riding (5\%)
    \item Swimming (5\%)
    \item Tennis (5\%)
    \item Hockey (10\%)
    \item Dancing (3\%)
    \item Judo (2\%)
  \end{itemize}
  \newpage
  \item What are your hobbies?
  The main hobbies are:
  \begin{itemize}
    \item Playing videogames
    \item Acting
    \item Studying foreign languages
    \item Reading books
    \item Singing
    \item Playing the guitar
    \item Dancing
  \end{itemize}
  \item Do you organise meetings with your friends outside? \\
  Who play team sports organises also outside meetings and matches with friends to play in the right fields.
  Who play individual sports prefers to go out and meet with friends and then maybe go for a walk in the centre of Lugano.
  \item Would you provide personal information to an application that organise meetings between children?
  In your opinion, is there any critical problem? \\
    Opinions are divided into two parts: about 65\% on a social network like Instragram do not have any problem to provide personal information, like name, surname and a profile pic. The 25\% would like to provide only
    name and surname and the remaining 10\%, on Instragram but using a fake name, would prefer to not provide any kind of personal information. The majority of them understood the big ethical issue which is the identity verification.
  \item In your opinion, is there a way to verify that the child that you are going to meet has really your own age? \\
    Without giving them any hint, someone suggested to divide in some way the age ranges. I told them that the most efficient way to resolve this is to base on the school, which can provide the children's user profiles.
    In this way only who has a profile which is provided by the belonging school (and so verified) can use the application. They found that the proposal is smart, because being sure that they are going to meet children of
    the same age (more or less) seems to be the major priority (and also for their parents).

  \item Can it works? \\
  They were excited and convinced about the proposal, expecially bacuase they can be sure about the age verification. \\
  \item Would you prefer to play at the park or at the football field or whatever with a child with exactly your own age or you may want also children younger/older than you? \\
  Almost all of them want to play with children of the same age or with children that are maximum 3 years youger/older that them.
  Only one of them clearly stated that she would play with older people (even more than she, like also 20 years old). Why? To improve! (she was talking about playing basketball). \\
  \item At the end, what do you think? \\
  The children gave me some new ideas:
  \begin{itemize}
    \item A feedback system about the partecipants to event and also about the event itself. Thus, I asked: what if this become a defamation tool? After this, about the 60\% of them told me that it would be better not to implement the feedback feature,
    whereas the other 40\% said that it would be a cool and useful feature only if the feedback is justified by an appropriate comment.
    \item A chat between people who are going to partecipate at the same event, which will be automatically opened when all the users subscribed to the event.
  \end{itemize}
\end{enumerate}

\end {document}
