\documentclass[12pt]{report}
\usepackage{graphicx}
\usepackage{hyperref}
\usepackage{fancyhdr}
\usepackage[export]{adjustbox}
\usepackage{geometry}
\usepackage[showframe]{geometry}

<<<<<<< HEAD
\begin{document}
\setlength{\voffset}{-1in}
\setlength{\oddsidemargin}{-60pt}
\begin{titlepage}
\vspace*{2pt}
\parbox[b]{0.75\textwidth[124pt]}{\includegraphics[width=130pt]{University-of-Lugano.png}}
\hspace*{10pt}
\rule{1pt}{\textheight}
\end{titlepage}
%\title {

%{\bf\Huge Concept Statement}
%\\\large Human-Computer Interaction
%\\\small Academic Year 2017/2018 \break
%\\\large \textbf{Group 7}
%\\large Alessandra Vicini \\ Edoardo Lunardi \\ Ozren Dabic \\ Pasquale Polverino \\ Paganoni Marco}






\newpage
\setlength{\voffset}{0pt}
\setlength{\oddsidemargin}{17pt}
\pagestyle{fancy}
\fancyhf{}
\rhead{Human Computer\\Interaction}
\chead{Spring \\Semester 2018}
\lhead{Group 7}
\cfoot{Name of the app}


For our group project, we have chosen to create an application that pairs up children who share the same
hobbies and interests. The main idea is to create a platform which keeps track of all meetings that are
taking place, so if someone wants to join, he/she can simply send a message to the other children who are
already playing or their parents. Then, since we have noticed that many children have made requests for a
plat- form on which they do not have to provide too much personal information about themselves to join in,
we have taught of a profile in which only basic information is required. Once logged in, they are divided
into similar age groups and they can look for someone who has the same interests in an intuitive way. To
solve the problem of how to make it secure and to be sure that nobody uses this plat- form except of the
children, we thought of using an authentication procedure done by the school. \\

\end {document}
=======
\title {
	\includegraphics[width = .2\linewidth]{University-of-Lugano.png} \break \break
	{\bf\Huge Concept Statement}
	\\\large Human-Computer Interaction
	\\\small Academic Year 2017/2018 \break
	\\\large \textbf{Group 7}}
	\author{
	\\\large Alessandra Vicini \\ Edoardo Lunardi \\ Ozren Dabic \\ Pasquale Polverino \\ Paganoni Marco}
\date{March 02, 2018}
\begin{document}
	\pagestyle{empty}
	\maketitle
	\section*{\huge Our concept}
		While observing the submissions of 2018's "Research and design competition",
		we noticed that there is a high demand for applications that allow children
	 	to make new friends outside of their educational environment. To meet these
	  demands, we have thought of creating an application that allows its user base
		to connect with new people based on parameters such as age, hobbies, interests
		and location. The main features of the app include, but aren't limited to:
	\begin{enumerate}
		\item Messaging system that allows for an easy exchange between those who wish
		 			to meet, complete with Safe For Work filters that censor bad language and
		  		penalise abusive behaviour;
		\item Account system with minimal dependency on private information;
		\item Account authentication provided by educational institutions that the
					users are associated with (i.e. schools, kindergartens,...);
		\item Two types of activity searching: a quick search that highlights activities
		 			of all types that are currently happening in the local area, or a filtered
		  		search, that categorises activities based on type, age of participants,
			 		location and the time in which it takes place.
	\end{enumerate}

	\end {document}
>>>>>>> d060c5480274d23003b3b42bdef2368e464ac94a
