
\documentclass[12pt]{article}
\usepackage{graphicx}
\usepackage{hyperref}

\title {
\includegraphics[width = .2\linewidth]{University-of-Lugano.png} \break \break
{\bf\Huge Concept Statement}
\\\large Human-Computer Interaction
\\\small Academic Year 2017/2018 \break
\\\large \textbf{Group 7}
\\\large Alessandra Vicini \\ Edoardo Lunardi \\ Ozren Dabic \\ Pasquale Polverino \\ Paganoni Marco}
\date{March 02 2018}

\begin{document}
\maketitle
\thispagestyle{empty}
\newpage
\thispagestyle{empty}
\subsection*{Our idea}

\noindent
  We have chosen to create an application that pairs up
  children who share the same hobbies and interests. The main idea is to create
  a platform which keeps track of all meetings
  that are taking place, so if someone wants to join, he/she can simply send a
  message to the other children who are already playing or their parents. Then,
  since we have noticed that many children have
  made requests for a plat- form on which they do not have to provide too much
  personal information about themselves to join in, we have taught of a profile
  in which only basic information is required.
  Once logged in, they are divided into similar age groups and they can look for
  someone who has the same interests in an intuitive way. To solve the problem
  of how to make it secure and to be sure that nobody uses this plat- form except
  of the children, we thought of using an authentication procedure done by the school. \\
\end {document}
