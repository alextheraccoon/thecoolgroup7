\documentclass[12pt]{article}
\usepackage{graphicx}
\usepackage{hyperref}

\title {
	\includegraphics[width = .2\linewidth]{University-of-Lugano.png} \break \break
	{\bf\Huge Concept Statement}
	\\\large Human-Computer Interaction
	\\\small Academic Year 2017/2018 \break
	\\\large \textbf{Group 7}
	\\\large Alessandra Vicini \\ Edoardo Lunardi \\ Ozren Dabic \\ Pasquale Polverino \\ Paganoni Marco}

\begin{document}
	
	\maketitle
	\newpage
	
	\part*{Our concept}
	While observing the submissions of 2018's "Research and design competition", we noticed that there is a high demand for applications that allow children to make new friends outside of their educational environment. To meet these demands, we have thought of creating an application that allows its user base to connect with new people based on parameters such as age, hobbies, interests and location. The main features of the app include, but aren't limited to:
	\begin{enumerate}
		\item Messaging system that allows for an easy exchange between those who wish to meet, complete with Safe For Work filters that censor bad language and penalise abusive behaviour;
		\item Account system with minimal dependency on private information;
		\item Account authentication provided by educational institutions that the users are associated with (i.e. schools, kindergartens,...);
		\item Two types of activity searching: a quick search that highlights activities of all types that are currently happening in the local area, or a filtered search, that categorises activities based on type, age of participants, location and the time in which it takes place. 
	\end{enumerate}
	
	\end {document}