\documentclass[12pt]{article}
\usepackage{graphicx}
\usepackage{hyperref}

\title {
\includegraphics[width = .2\linewidth]{University-of-Lugano.png} \break \break
{\bf\Huge Concept Statement}
\\\large Human-Computer Interaction
\\\small Academic Year 2017/2018 \break
\\\large \textbf{Group 7}
\\\large Alessandra Vicini \\ Edoardo Lunardi \\ Ozren Dabic \\ Pasquale Polverino \\ Paganoni Marco}
\date{March 02 2018}

\begin{document}
\maketitle
\thispagestyle{empty}

\newpage
\setcounter{page}{1}

\section*{Questions}
Ciao ragazzi, mi chiamo Edoardo Lunardi e sono al primo anno di Informatica all'USI.
\begin{enumerate}

  \item Sapete perche' sono qui oggi? \\
  Siamo stati incaricati di raccogliere alcune delle vostre che avete proposto per il progetto Interaction Design and Children 2018:
  praticamente i ragazzi e bambini di tutto il mondo sono invitati a proporre le loro idee sulle future tecnologie nell'ambito social.
  Vorrei quindi chiedervi:
  \item Come e quanto usate il vostro smartphone?
  \item Fate sport? Quali?
  \item Quali sono i vostri hobby?
  \item Organizzate incontri con i vostri amici per attivita all'aperto? \\
  Abbiamo raccolto varie vostre idee e abbiamo notato che hanno tutte un aspetto in comune: fare nuove conoscenze. Abbiamo quindi pensato
  a una sorta di applicazione per il vostro smartphone in grado di organizzare degli incontri tra di voi. Vi spiego meglio. Gli utenti si
  registrano, inseriscono i loro sport e hobby preferiti. Successivamente possono segnalare l'intenzione di volere giocare a una partita di
  calcio, basket, hockey o qualsiasi sport di squadra, oppure di voler incontrarsi a parco a giocare a qualsiasi gioco. L'applicazione proporra' poi
  tutti i luoghi idonei, l'utente sceglie il luogo insieme all'orario preferito e il gioco è fatto. Chi ha cercato "l'evento" ovviamente sara' a conoscenza
  di quante persone verranno, quando verranno e, cosa piu' importante, chi verra' insieme ad altre informazioni.
  \item Vi fidereste a dare i vostri dati personali ad una applicazione come la
    nostra?
  \item Secondo voi c'e' un modo affidabile per verificare che il ragazzino che
    andrete a incontrare sia un vostro coetano?
  Abbiamo infatti pensato che e' fondamentale anche per i vostri genitori essere sicuro delle persone che andrete a incontrare.
  Siamo giunti alla conclusione che l'unico modo per essere sicuri che l'utente Mario Rossi sia veramente un vostro coetaneo e' quello di appoggiarsi alla scuola.
  Vi spiego meglio. Ogni ragazzo avra' un account gia' creato dalla scuola. Pensiamo che la scuola sia un ente affidabile, infatti la scuola sa che tu sei Mario Rossi,
  hai 13 anni e frequenti la 2A di questo istituto. In questo modo la scuola fornisce i dati per entrare nel proprio account e nessun'altro puo' usarlo a parte voi.
  \item Secondo voi puo' funzionare? Vi sembra un modo affidabile?
  \item L'ultima domanda che vorrei farvi e' la seguente: preferireste giocare al parco o al campo di calcio o quello che sia con un vostro coetaneo o con un ragazzo piu'
  grande/piu' piccolo di voi?
  \item Cosa ne pensate? Grazie per la disponibilita'.
\end{enumerate}

\end {document}
